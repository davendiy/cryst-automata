\documentclass[12pt]{article}
\usepackage{a4wide}
\usepackage{setspace}
\usepackage{amsmath,amssymb,amsthm}
\usepackage{graphicx}

\usepackage[english]{babel}

\usepackage{hyperref}
\hypersetup{
 colorlinks=true,
 linkcolor=blue,
 filecolor=magenta,      
 urlcolor=cyan,
 pdftitle={Overleaf Example},
 pdfpagemode=FullScreen,
}

\DeclareMathOperator{\tr}{tr}
\DeclareMathOperator{\Dom}{Dom}
\newtheorem{theorem}{Theorem}
\newtheorem{lemma}{Lemma}

\newtheorem{proposition}[theorem]{Proposition}
\newtheorem{corollary}{Corollary}[theorem]
\newtheorem{conjecture}{Conjecture}

\theoremstyle{definition}
\newtheorem{definition}{Definition}
\newtheorem{example}{Example}
\newtheorem{problem}{Problem}
\newtheorem{question}{Question}
\newtheorem{remark}{Remark}
\newtheorem{algorithm}{Scheme}

% \graphicspath{ {./assets} }


\title{Planar Groups}
\author{ Davyd Zashkolnyi }


\begin{document}
\maketitle
\begin{abstract}
  A group $G$ admits self-replicating action of degree $d$ if it acts on a space of words over alphabet of size $d$.
  A group $G$ admits .... if and only if it has a subgroup $H$ of index $d$ and a homomorphism
  $\phi : H \rightarrow G$ with trivial core. Every crystallographic group admits a self-replicating action.
  We determine self-replicating degrees for crystallographic groups of dimension $n=2,3$.
\end{abstract}

\tableofcontents

\newpage
\section{Self-similar actions}

\textbf{..........motivation...........}

Let $X$ be a finite set (alphabet) and $X^{*}$ the set of all words over $X$, including the empty word $\epsilon$.
The set $X^{*}$ with operation concatenation of words is a free monoid generated by $X$.
The length of a word $v$ is denoted by $|v|$. The set $X^{*}$ can be also viewed as a $d$-regular rooted tree with $\epsilon$
being a root and $d=|X|$ being the size of the alphabet.

\begin{definition}
  A faithful action of a group $G$ on the set $X^{*}$ is called \textit{self-similar} if for every $g\in G$
  and $x\in X$ there exist $y\in X$ and $h\in G$ such that $g(xw)=yh(w)$ for all $w\in X^{*}$. A group $G$ is
  called \textit{self-similar} if it admits a self-similar action over some alphabet $X$.
\end{definition}

\begin{definition}
  A self-similar action is called \textit{self-replicating} or \textit{recurrent} if for every triple
  $g \in G, x \in X$ and $h \in G$ there exist $y \in X$ such that $g(xw) = yh(w)$.
\end{definition}

Informally, a group is self-similar if its action repeats itself on some subset. Any element $g \in G$ can be defined
accordingly to it's action on $X^*$. Denote $g = \pi (g_1 g_2 \dots g_d)$ for $\pi \in Aut(X)$ and $g_i \in G$. The
element $g$ acts on a word $w \in X^*$ according to its prefix $x_i \in X$:
$$g(x_i w) = \pi(x_i) g_i(w).$$
Here the element $g_i$ is called a \textit{restriction} and is denoted as $g|_{x} = g_i$ for $x = x_i$.
We can also define right multiplication rule $(f * g)(x) = f(g(x))$: 
\[a = \pi (a_1, a_2, \dots a_n), \quad b = \sigma (b_1, b_2, \dots b_n)\]
\[a * b = \pi * \sigma (a_{\sigma(1)}*b,\, a_{\sigma(2)}*b,\, \dots \,
a_{\sigma(n)}*b)\]

To study self-similar groups, a convenient toolkit was invented by Nekrashevych in \cite{Nekrashevych:Self-similar},
that is, virtual endomorphisms.

\begin{definition}
  A \textit{virtual endomorphism} $\phi$ of a group $G$ is a homomorphism from its subgroup of
  finite index $\Dom \phi := H < G$.

  A virtual endomorphism $\phi$ is called \textit{simple} if $G$ has no $\phi$-invariant normal subgroups.
\end{definition}

\begin{theorem}(Nekrashevych)
  Every self-replicating action induces a surjective associative virtual endomorphism $\phi_x (g) = g|_x$ for any $x \in X$.
  The index of the domain $[G : \Dom \phi_x]$ equals to size of the alphabet $d=|X|$.

  Every simple surjective virtual endomorphism $\phi : H \rightarrow G$ forms a faithful self-replicating
  action over an alphabet $X = G/H = \{x_i\}$:
  \[
  g \cdot x_i = x_j \cdot \phi(g_j^{-1}gg_i),
  \]
  where $x_i = g_i$ are the right coset representatives of $H$ in $G$, and $j$ is chosen in a way that $g_j^{-1}g g_i \in H$.
\end{theorem}

Virtual endomorphisms were used to analyze self-similar actions for a wide range of groups: free abelian groups,
finitely generated nilpotent groups, solvable groups, wreath products of abelian groups ....

\begin{theorem}
  Every finitely generated virtually abelian group admits a self-similar action. In particular, every crystallographic
  group admits a self-similar action.
\end{theorem}

\begin{proof}
  A finitely generated virtually abelian group $G$ contains $H = \mathbb{Z}^n$ for some $n\in\mathbb{N}$ as a subgroup of
  finite index, which admits a self-similar action by \cite{Nekrashevych:Self-similar}.

  Let $(H,X^{*})$ be a self-similar action. Let $D$ be a set of coset representatives for $H$ in $G$ with $e\in D$.
  We construct a self-similar action of $G$ over the alphabet $Y=D\times X$. The action is defined by
  the rule: for $g\in G$ and $(d,x)\in Y$,
  \[
  g(d,x)=(c,y) \quad \mbox{ and } \quad g|_{(d,x)}=h|_x,
  \]
  where $c\in D$ is the unique element such that $h=c^{-1}gd\in H$ and $y=h(x)$.
  It is straightforward to check that the action is well-defined. This action is faithful,
  because every $g\in G\setminus H$ acts nontrivially on $Y$ and $H$ acts faithfully on the subspace $(\{e\}\times X)^{*}$ of $Y^{*}$.

  \begin{align*} % the action is well-defined
    g_1g_2 * (d,x) &= g_1*(c,h_2(x))* h_2|_x \qquad && h_2=c^{-1}g_2d\\
    &= (b,h_1(h_2(x))) *h_1|_{h_2(x)}h_2|_x \qquad && h_1=b^{-1}g_1c\\
    (g_1g_2) *(d,x) &= (b,h(x))*h|_x \qquad && h=b^{-1}g_1g_2d=h_1h_2
  \end{align*}

\end{proof}

Despite every virtual abelian group is self-similar, not every one admits self-replicating action.
As an example, consider $G = \mathbb{Z}_2 \times \mathbb{Z}$ which is not
self-replicating due to \cite{Sidki}. We will prove in the next section that every crystallographic group is self-replicating.


% =====================================================================================================================


\section{Crystallographic groups} 

Let $\mathbf{A}(n)$ be a group of affine transformations and $\mathbf{O}(n)$ a group of isometries of
$\mathbb{R}^n$ respectively.

\begin{definition}
  A \textit{crystallographic group} is a discrete cocompact subgroup of $\mathbf{O}(n)$. It is also called
  a \textit{plane group} for the case $n=2$ and a \textit{space group} for the case $n=3$.
\end{definition}

\begin{theorem}[Bieberbach, 1910]
\begin{enumerate}
  \item If $G \subset \mathbf{A}(n)$ is a crystallographic group, then the set of translations
  $G \cap (I \times \mathbb{R}^n)$ is a torsion free and finitely generated abelian group of rank $n$, and
  is a maximal abelian and normal subgroup of finite index.

  \item Two crystallographic groups of dimension $n$ are isomorphic if and only if they are conjugate
  in the affine group $\mathbf{A}(n)$.

  \item For any $n\in\mathbb{N}$ there are only finite number of isomorphism classes of crystallographic groups of dimension $n$.
\end{enumerate}
\end{theorem}

\begin{theorem}[Zassenhaus]
  A group $G$ is isomorphic to a crystallographic group of dimension $n$ if and only if $G$ has a normal,
  free abelian subgroup $\mathbb{Z}^n$ of finite index, which is a maximal abelian subgroup of $G$.
\end{theorem}

A crystallographic group $G$ can be uniquely defined via triplet $(P, L, \alpha)$ of a point group
$P < GL_n(\mathbb{R})$, a lattice $L \subset \mathbb{R}^n$ such that there exists short exact sequence
$L \rightarrow G \rightarrow P$ and a map $\alpha : P \rightarrow \mathbb{R}^n$ such that
$G = \{ (A, \alpha(A) + t) \,|\, A \in P, \, t \in L \}$.

By the Bieberbach theorem, $L$ is the maximal normal free
abelian subgroup of $G$. The image of $\alpha$ belongs to $\mathbb{R}^n / L$ and is called System of Non-trivial
Translations (SNoT). Simply, these are translations which make $G$ different from a semi-direct product $P \ltimes L$.

\begin{definition}
  A crystallographic group which is isomorphic to a semi-direct product $P \ltimes L$ of its point group and lattice is
  called \textit{symmorphic}. 
\end{definition}

\begin{theorem}(Bondarenko)
  A surjective virtual endomorphism of crystallographic group is an isomorphism from its subgroup to itself.
\end{theorem}

\begin{corollary}
  For every surjective virtual endomorphism $\phi : H \rightarrow G$ of a crystallographic group $G$ there exists an
  affine element $(A, t) \in \mathbf{A}(n)$ such that $\phi(g) = (A, t)^{-1}g(A, t)$ and $H = \phi^{-1}(G) = (A, t)G(A, t)^{-1}$.
\end{corollary}

Consider a crystallographic group $G = (P, L, \alpha)$, its isomorphic subgroup of finite index $H = (P', L', \alpha')$
and a virtual endomorphism $\phi : H \rightarrow G$ which is given by $\phi(g) = (A, t)^{-1}g(A, t)$.
Its action on the generators of $L$ is reduced to linear operator:
$$
(A, t)(E, e_i)(A, t)^{-1} = (E, Ae_i + t - AA^{-1}t) = (E, At).
$$
From this follows that $L' = A^{-1}L < L$ and its index equals to the $\det(A)$. 

\begin{lemma}
  $H = (P, L', \alpha)$ and $[G:H] = [L:L']$.
\end{lemma}

\begin{theorem}
  A surjective virtual endomorpism $\phi : H \rightarrow G$ given by $\phi(g) = (A, t)^{-1}g(A, t)$ is simple if and
   only if characteristic polynomial of $A$ isn't divisible by a monic polynomial with integral coefficients. 
\end{theorem}

From now on we consider $L$ to be $\mathbb{Z}^n$, by using generators of original $L$ as a basis. As a result, the SNoT
consists of vectors from $[0,1]^n$. We now can make an upper bound for self-replicating degree of crystallographic groups: 

\begin{theorem}
  Every crystallographic group $G$ admits a self-replicating action over an alphabet of size $(m+1)^n$ where $m$ is
  the least common multiplier of denominators of entities of $\alpha(A)$ for every $A \in P$ in the basis of $L$.
\end{theorem}

\begin{corollary}
  Every planar group admits a self-replicating action over an alphabet of size $\le 9$. 
\end{corollary}

\begin{corollary}
  Every symmorphic crystallographic group admits a self-replicating action over an alphabet of size $\le 2^n$.
\end{corollary}

Consider the planar case $G = (P, \mathbb{Z}^2, \alpha)$
Let $(A, t)$ be an affine element which induces surjective virtual endomorphism
$\phi : H \rightarrow G, \, \phi(g) = (A, t)^{-1}g(A, t)$, where $H = (A, t) G (A, t)^{-1} \simeq G$.
From surjectivity follows that $A$ has integral entities and $|\det(A)| > 1$. Denote 
$$A = \left(\begin{array}{rr}
x_{0} & x_{1} \\
x_{2} & x_{3}
\end{array}\right).$$


\begin{proposition}
A surjective virtual endomorphism, generated by $(A, t)$, is simple if and only if
$\det{A} \pm \tr{A} \neq -1$.
\end{proposition}

\begin{proof}
$A^{-1}$ is simple if and only if its characteristic polynomial $f(x)$ isn't divisible by a monic polynomial with
integral coefficients. For $\dim =2$ it means that $f(x)$ has no integral coefficient and no integral root. $A$ has an
eigenvalue $\lambda$ if and only if $1/\lambda$ is an eigenvalue of $A^{-1}$. Let $\lambda_1, \lambda_2$ be two
eigenvalues of $A$. We can write an explicit form of $f(x)$: 
$$f(x) = x^2 - \left(\frac{1}{\lambda_1} + \frac{1}{\lambda_2}\right)x + \dfrac{1}{\lambda_1 \lambda_2} =
x^2 - \dfrac{\lambda_1 + \lambda_2}{\lambda_1 \lambda_2}x + \frac{1}{\lambda_1 \lambda_2} =
x^2 - \dfrac{\lambda_1 + \lambda_2}{\det(A)}x + \frac{1}{\det(A)}$$

From this follows, that $f(x)$ have at least one rational coefficient for $|\det(A)| >1$. 
Let's now consider $h(x)$ the characteristic polynomial of $A$. It's explicit form: 
$$h(x) = x^2 - (\lambda_1 + \lambda_2)x + \lambda_1\lambda_2 = x^2 + dx + \det(A).$$
$A$ has integral entities, therefore $h(x)$ has integral coefficients as well. By rational root theorem, $p/q$
is a root of integral polynomial if and only if $q$ is an integer factor of the leading coefficient. $h(x)$ is monic
as a characteristic polynomial. As a result, every rational eigenvalue of $A$ is integral, so the only possible
integral eigenvalue of $A^{-1}$ is $\pm1$. 

Finally, let's write conditions on the coefficients:
$$h(\pm1) = 1  \mp (x_0 + x_3) + (x_0x_3 - x_1x_2) \neq 0$$
$$\det(A) \pm(x_0 + x_3) \neq -1$$
\end{proof}


We now formulate some computational problems: given a crystalographic group $G = (P, L, \alpha)$

\begin{enumerate}

  \item (self-covering degrees) decide whether there exists a subgroup of index $d$ that is isomorphic to $G$
  $$SCDegrees = \{d \in \mathbb{N} \, | \, G \text{ has isomorphic subgroup of index d } \};$$

  \item (self-replicating degrees) decide whether there exists a self-replicating action of $G$ over an
  alphabet of size $d \in \mathbb{N}$
  $$SRDegrees(G) = \{d \in \mathbb{N}\, | \, d \text{ is a self-replicating degree of } G\};$$

  \item (min self-replicating degree) find the minimal self-replicating degree of the group $G$
  $$MinSRDegree(G) = \min (SRDegrees(G));$$

\end{enumerate}

Now let's describe an algorithm to generate sets $SCDegrees$ and $SRDegrees$. Given a crystallographic
group $G = (P, \mathbb{Z}^n, \alpha)$: 

\begin{enumerate}

 \item compute the structure of integer matrices normalizing the group $P$. The set of solutions is a disjoint union of
 matrices of finitely many types corresponding to $\sigma \in Aut(P)$. For every automorphism $\sigma \in Aut(P)$ we
 solve system of linear equations $XB = \sigma(B)X, \, B \in P$ over integers. Every solution $X_\sigma$ of type $\sigma$
 is an integer linear combination of integral matrices that consists of variables $x_i \in \mathbb{Z}$. 

 \item ($SCDegrees$) determine conditions on the coefficients of matrices $A_{\sigma}$ to preserve cocyle.
 For every type $\sigma$ we solve system of congruences

 $(A_\sigma, t)(B, \alpha(B))(A_\sigma, t)^{-1} = (\sigma(B), \alpha(\sigma(B))) \mod \mathbb{Z}^n$ for every  $B \in P$;

 we can extend the left part:

 $(A, t)(B, \alpha(B))(A^{-1}, -A^{-1}t) = (ABA^{-1}, -ABA^{-1}t + A \alpha(B) + t) $

 $= (ABA^{-1}, (E - ABA^{-1})t + A \alpha(B)) = (ABA^{-1}, \alpha(ABA^{-1})) \mod \mathbb{Z}^n$

 or

 \begin{equation}\label{eq:congruence}
   A_\sigma \alpha(B) - \alpha(\sigma(B)) = (\sigma(B) - E)t  \mod \mathbb{Z}^n
 \end{equation}
 
 This system of equations is linear over an abelian group $\mathbb{Q}/\mathbb{Z}$.

\begin{theorem}
  The system (\ref{eq:congruence}) can be solved algorithmically.
\end{theorem}
\begin{proof}
\end{proof}
 
 \item ($SRDegrees$) solve the equations
 \begin{equation}\label{eq:det}
 |\det{A}| \neq 1
 \end{equation}
 and
 \begin{equation}\label{eq:eigenvalues}
 \det{A} \pm \tr{A} \neq -1
 \end{equation}
 to exclude combinations that form virtual endomorphisms that aren't simple.

 \begin{theorem}
  The equations (\ref{eq:det}) and (\ref{eq:eigenvalues}) can be solved algorithmically for dim=2.
 \end{theorem}
 \begin{proof}
 \end{proof}
 
\end{enumerate}

\newpage
\subsection{ Group 7 p2mg }


The Group 7 ($p2mg$) from ITA is generated by the following elements: 
$$G_7 = \left\langle\left(\begin{array}{rr|r}
1 & 0 & 1 \\
0 & 1 & 0 \\
\end{array}\right),\left(\begin{array}{rr|r}
-1 & 0 & 1/2 \\
0 & 1 & 0 \\
\end{array}\right)\left(\begin{array}{rr|r}
-1 & 0 & 0 \\
0 & -1 & 0 \\
\end{array}\right)  \right\rangle = \langle a,b,c \rangle$$
The normalizer of the point group consists of matrices of two types: 
$$A_1, A_2 = \left(\begin{array}{rr}
0 & x_{1} \\
x_{0} & 0
\end{array}\right), \left(\begin{array}{rr}
x_{0} & 0 \\
0 & x_{1}
\end{array} \right) \quad   \,  x_0,x_1 \in \mathbb{Q}^*$$

Solving the system of congruences from 2) we obtain that for $A_1$ system doesn't have solutions.

For the $A_2$ we get that $x_0$ should be odd integral, and $x_1$ integral. 

All the conjugations are generated by
$$
(A, t) = \left(\begin{array}{rr|r}
x_0 & 0 & a_0 \\
0 & x_1 & a_1 \\
\hline
0 & 0 & 1
\end{array}\right)
$$
where $x_0 = 2k-1, x_1 \in Z, a_0 a_1 \in 1/2 Z$. 

Therefore, $$SCDegrees = \{ |\det(A_2)| = |x_0 x_1|, \, |\,  x_0 \in 2\mathbb{Z} + 1, \, x_1 \in \mathbb{Z}\} = \mathbb{N}$$
The inverse of matrix $A_2$ fits the condition 3) if and only if $|x_0| > 1, |x_1| > 1$. Therefore, 

$$SRDegrees = \{ (2k + 1)(m+1) \, | \, k,m \in \mathbb{N}\}$$
or not primes and not powers of 2. In particular, the minimal self-replicating degree is 6 when $x_0 = 3, x_1 = 2$.

\subsection{Group 13}

The Group 13 (p3) from ITA is generated by the following elements:
$$G_{13} = \left\langle\left(\begin{array}{rrr}
0 & -1 & 0 \\
1 & -1 & 0 \\
0 & 0 & 1
\end{array}\right), \left(\begin{array}{rrr}
1 & 0 & 1 \\
0 & 1 & 0 \\
0 & 0 & 1
\end{array}\right), \left(\begin{array}{rrr}
1 & 0 & 0 \\
0 & 1 & 1 \\
0 & 0 & 1
\end{array}\right)\right\rangle$$
The normalizer of the point group consists of matrices of two types:
$$A_1, A_2 = \left(\begin{array}{rr}
-x_{0} & x_{0} + x_{1} \\
x_{1} & x_{0}
\end{array}\right), \left(\begin{array}{rr}
x_{0} + x_{1} & -x_{0} \\
x_{0} & x_{1}
\end{array}\right), \quad x_0,x_1 \in \mathbb{Q}^*$$
The group $G_{13}$ is symmorphic, i.e. $G_{13} = P \ltimes \mathbb{Z}^2$, from which follows that congruences from 2)
are trivial and the only condition for translation is to be integral.

$$SCDegrees = \{ |\det(A_1)|, |\det(A_2)|\} = \{ |x_{0}^{2} + {\left(x_{0} + x_{1}\right)}| \quad | \quad x_0, x_1 \in \mathbb{Z}\}$$

\subsubsection{Self-replicating degrees.}
\begin{enumerate}
\item Consider the first matrix
$$A = \left(\begin{array}{rr}
x_{0} + x_{1} & -x_{0} \\
x_{0} & x_{1}
\end{array}\right)$$

For the determinanat we have $\det{A} = x_{0}^{2} + {\left(x_{0} + x_{1}\right)} x_{1} \in \mathbb{Z}$.
The simplicity of the respective virtual endomorphism depends on the following diophantine equations:
$$x_{0}^{2} + {\left(x_{0} + x_{1}\right)} x_{1} \neq \pm 1$$
$$x_{0}^{2} + {\left(x_{0} + 2\right)} x_{1} + x_{1}^{2} + x_{0} \neq -1$$
$$x_{0}^{2} + {\left(x_{0} - 2\right)} x_{1} + x_{1}^{2} - x_{0} \neq -1$$

Solving the first equation on the determinant $\det{A} \neq \pm 1$ we obtain the pairs:
$$\left[\left(0, 1\right), \left(-1, 1\right), \left(1, -1\right), \left(-1, 0\right), \left(1, 0\right), \left(0, -1\right)\right]$$

Solving next equations on the eigenvalues $\det{A} \pm \tr{A} = -1$:
$$\left[\left(0, -1\right), \left(0, 1\right)\right]$$

\item Consider the second matrix
$$A = \left(\begin{array}{rr}
-x_{1} & x_{0} + x_{1} \\
x_{0} & x_{1}
\end{array}\right)$$
Determinant: $\det{A} = -{\left(x_{0} + x_{1}\right)} x_{0} - x_{1}^{2} \in \mathbb{Z}$.
Self-replicating degrees:
$$-{\left(x_{0} + x_{1}\right)} x_{0} - x_{1}^{2} \neq \pm 1$$
$$-x_{0}^{2} - x_{0} x_{1} - x_{1}^{2} \neq -1$$
$$-x_{0}^{2} - x_{0} x_{1} - x_{1}^{2} \neq -1$$
Solving diophantine equation for the determinant $\det{A} = \pm 1$:
$$\left[\left(0, 1\right), \left(-1, 1\right), \left(1, -1\right), \left(-1, 0\right), \left(1, 0\right), \left(0, -1\right)\right]$$
Solve diophantine equation for eigenvales:
$$\left[\left(0, 1\right), \left(-1, 1\right), \left(1, -1\right), \left(-1, 0\right), \left(1, 0\right), \left(0, -1\right)\right]$$
$$\left[\left(0, 1\right), \left(-1, 1\right), \left(1, -1\right), \left(-1, 0\right), \left(1, 0\right), \left(0, -1\right)\right]$$
\end{enumerate}


\newpage
\begin{scriptsize}
\begin{tabular}{|l|l|l|l|} \hline

matrix & det & eigenvalue != 1 & eigenvalue != -1 \\ \hline \hline
$\left(\begin{array}{rr}
x_{0} + x_{1} & -x_{0} \\
x_{0} & x_{1}
\end{array}\right)$ & $x_{0}^{2} + {\left(x_{0} + x_{1}\right)} x_{1}$ & $x_{0}^{2} + {\left(x_{0} + 2\right)} x_{1} + x_{1}^{2} + x_{0} + 1$ & $x_{0}^{2} + {\left(x_{0} - 2\right)} x_{1} + x_{1}^{2} - x_{0} + 1$ \\ \hline
$\left(\begin{array}{rr}
-x_{1} & x_{0} + x_{1} \\
x_{0} & x_{1}
\end{array}\right)$ & $-{\left(x_{0} + x_{1}\right)} x_{0} - x_{1}^{2}$ & $-x_{0}^{2} - x_{0} x_{1} - x_{1}^{2} + 1$ & $-x_{0}^{2} - x_{0} x_{1} - x_{1}^{2} + 1$ \\ \hline
\end{tabular}
\\
\begin{tabular}{|l|l|l|} \hline
matrix & det & except \\ \hline \hline
$\left(\begin{array}{rr}
x_{0} + x_{1} & -x_{0} \\
x_{0} & x_{1}
\end{array}\right)$ & $x_{0}^{2} + {\left(x_{0} + x_{1}\right)} x_{1}$ & $\left\{\left(-1, 0\right),\left(-1, 1\right),\left(0, -1\right),\left(0, 1\right),\left(1, -1\right),\left(1, 0\right)\right\}$ \\ \hline
$\left(\begin{array}{rr}
-x_{1} & x_{0} + x_{1} \\
x_{0} & x_{1}
\end{array}\right)$ & $-{\left(x_{0} + x_{1}\right)} x_{0} - x_{1}^{2}$ & $\left\{\left(-1, 0\right),\left(-1, 1\right),\left(0, -1\right),\left(0, 1\right),\left(1, -1\right),\left(1, 0\right)\right\}$ \\ \hline
\end{tabular}

\end{scriptsize}


\newpage
As a result, we run the proposed algorithm for every planar group and get the following self-repicating degrees.
For shorter notation, we use $OZ := \{2k + 1 | k \in \mathbb{Z}\}$ i.e. odd integers.

\begin{scriptsize}

\begin{tabular}{|l|l|} \hline
num\footnotemark[1]{} & srdegrees\footnotemark[2]{} \\ \hline \hline
$3$ & $\begin{tabular}{l}
$\left\{ x_{0} x_{1} \, | \, (x_{0},x_{1})\in \mathbb{Z} \times \mathbb{Z} \, / \,\left\{\left(n_{1}, -1\right),\left(n_{1}, 1\right),\left(-1, -1\right),\left(-1, 1\right),\left(-1, n_{1}\right),\left(1, -1\right),\left(1, 1\right),\left(1, n_{1}\right)\right\}\right\}$ \\
\end{tabular}$ \\ \hline
$4$ & $\begin{tabular}{l}
$\left\{ x_{0} x_{1} \, | \, (x_{0},x_{1})\in OZ \times \mathbb{Z} \, / \,\left\{\left(n_{1}, -1\right),\left(n_{1}, 1\right),\left(-1, -1\right),\left(-1, 1\right),\left(-1, n_{1}\right),\left(1, -1\right),\left(1, 1\right),\left(1, n_{1}\right)\right\}\right\}$ \\
\end{tabular}$ \\ \hline
$5$ & $\begin{tabular}{l}
$\left\{ {\left(x_{0} - 2 \, x_{1}\right)} x_{0} \, | \, (x_{0},x_{1})\in \mathbb{Z} \times \mathbb{Z} \, / \,\left\{\left(-1, -1\right),\left(1, 1\right),\left(2 \, t_{0} + 1, t_{0}\right),\left(-1, 0\right),\left(-1, n_{1}\right),\left(2 \, t_{0} - 1, t_{0}\right),\left(1, 0\right),\left(1, n_{1}\right)\right\}\right\}$ \\
\end{tabular}$ \\ \hline
$6$ & $\begin{tabular}{l}
$\left\{ -x_{0} x_{1} \, | \, (x_{0},x_{1})\in \mathbb{Z} \times \mathbb{Z} \, / \,\left\{\left(-1, -1\right),\left(-1, 1\right),\left(1, -1\right),\left(1, 1\right)\right\}\right\}$ \\
$\left\{ x_{0} x_{1} \, | \, (x_{0},x_{1})\in \mathbb{Z} \times \mathbb{Z} \, / \,\left\{\left(n_{1}, -1\right),\left(n_{1}, 1\right),\left(-1, -1\right),\left(-1, 1\right),\left(-1, n_{1}\right),\left(1, -1\right),\left(1, 1\right),\left(1, n_{1}\right)\right\}\right\}$ \\
\end{tabular}$ \\ \hline
$7$ & $\begin{tabular}{l}
$\left\{ x_{0} x_{1} \, | \, (x_{0},x_{1})\in \mathbb{Z} \times OZ \, / \,\left\{\left(n_{1}, -1\right),\left(n_{1}, 1\right),\left(-1, -1\right),\left(-1, 1\right),\left(-1, n_{1}\right),\left(1, -1\right),\left(1, 1\right),\left(1, n_{1}\right)\right\}\right\}$ \\
\end{tabular}$ \\ \hline
$8$ & $\begin{tabular}{l}
$\left\{ -x_{0} x_{1} \, | \, (x_{0},x_{1})\in OZ \times OZ \, / \,\left\{\left(-1, -1\right),\left(-1, 1\right),\left(1, -1\right),\left(1, 1\right)\right\}\right\}$ \\
$\left\{ x_{0} x_{1} \, | \, (x_{0},x_{1})\in OZ \times OZ \, / \,\left\{\left(n_{1}, -1\right),\left(n_{1}, 1\right),\left(-1, -1\right),\left(-1, 1\right),\left(-1, n_{1}\right),\left(1, -1\right),\left(1, 1\right),\left(1, n_{1}\right)\right\}\right\}$ \\
\end{tabular}$ \\ \hline
$9$ & $\begin{tabular}{l}
$\left\{ -x_{0}^{2} + 2 \, x_{0} x_{1} \, | \, (x_{0},x_{1})\in \mathbb{Z} \times \mathbb{Z} \, / \,\left\{\left(-1, -1\right),\left(-1, 0\right),\left(1, 0\right),\left(1, 1\right)\right\}\right\}$ \\
$\left\{ {\left(x_{0} - 2 \, x_{1}\right)} x_{0} \, | \, (x_{0},x_{1})\in \mathbb{Z} \times \mathbb{Z} \, / \,\left\{\left(-1, -1\right),\left(1, 1\right),\left(2 \, t_{0} + 1, t_{0}\right),\left(-1, 0\right),\left(-1, n_{1}\right),\left(2 \, t_{0} - 1, t_{0}\right),\left(1, 0\right),\left(1, n_{1}\right)\right\}\right\}$ \\
\end{tabular}$ \\ \hline
$10$ & $\begin{tabular}{l}
$\left\{ x_{0}^{2} + x_{1}^{2} \, | \, (x_{0},x_{1})\in \mathbb{Z} \times \mathbb{Z} \, / \,\left\{\left(-1, 0\right),\left(0, -1\right),\left(0, 1\right),\left(1, 0\right)\right\}\right\}$ \\
$\left\{ -x_{0}^{2} - x_{1}^{2} \, | \, (x_{0},x_{1})\in \mathbb{Z} \times \mathbb{Z} \, / \,\left\{\left(-1, 0\right),\left(0, -1\right),\left(0, 1\right),\left(1, 0\right)\right\}\right\}$ \\
\end{tabular}$ \\ \hline
$11$ & $\begin{tabular}{l}
$\left\{ x_{0}^{2} \, | \, x_{0}\in \mathbb{Z} \, / \,\left\{-1,1\right\}\right\}$ \\
$ \left\{ 2 \, x_{0}^{2} \, | \, x_{0}\in \mathbb{Z} \right\} $ \\
$ \left\{ -2 \, x_{0}^{2} \, | \, x_{0}\in \mathbb{Z} \right\} $ \\
$\left\{ -x_{0}^{2} \, | \, x_{0}\in \mathbb{Z} \, / \,\left\{-1,1\right\}\right\}$ \\
\end{tabular}$ \\ \hline
$12$ & $\begin{tabular}{l}
$ \left\{ -2 \, x_{0}^{2} \, | \, x_{0}\in \mathbb{Z} \right\} $ \\
$ \left\{ 2 \, x_{0}^{2} \, | \, x_{0}\in \mathbb{Z} \right\} $ \\
$\left\{ x_{0}^{2} \, | \, x_{0}\in OZ \, / \,\left\{-1,1\right\}\right\}$ \\
$\left\{ -x_{0}^{2} \, | \, x_{0}\in OZ \, / \,\left\{-1,1\right\}\right\}$ \\
\end{tabular}$ \\ \hline
$13$ & $\begin{tabular}{l}
$\left\{ -{\left(x_{0} + x_{1}\right)} x_{0} - x_{1}^{2} \, | \, (x_{0},x_{1})\in \mathbb{Z} \times \mathbb{Z} \, / \,\left\{\left(-1, 0\right),\left(-1, 1\right),\left(0, -1\right),\left(0, 1\right),\left(1, -1\right),\left(1, 0\right)\right\}\right\}$ \\
$\left\{ x_{0}^{2} + {\left(x_{0} + x_{1}\right)} x_{1} \, | \, (x_{0},x_{1})\in \mathbb{Z} \times \mathbb{Z} \, / \,\left\{\left(-1, 0\right),\left(-1, 1\right),\left(0, -1\right),\left(0, 1\right),\left(1, -1\right),\left(1, 0\right)\right\}\right\}$ \\
\end{tabular}$ \\ \hline
$14$ & $\begin{tabular}{l}
$\left\{ -x_{0}^{2} \, | \, x_{0}\in \mathbb{Z} \, / \,\left\{-1,1\right\}\right\}$ \\
$\left\{ x_{0}^{2} \, | \, x_{0}\in \mathbb{Z} \, / \,\left\{-1,1\right\}\right\}$ \\
\end{tabular}$ \\ \hline
$15$ & $\begin{tabular}{l}
$\left\{ -x_{0}^{2} \, | \, x_{0}\in \mathbb{Z} \, / \,\left\{-1,1\right\}\right\}$ \\
$\left\{ x_{0}^{2} \, | \, x_{0}\in \mathbb{Z} \, / \,\left\{-1,1\right\}\right\}$ \\
\end{tabular}$ \\ \hline
$16$ & $\begin{tabular}{l}
$\left\{ -{\left(x_{0} + x_{1}\right)} x_{0} - x_{1}^{2} \, | \, (x_{0},x_{1})\in \mathbb{Z} \times \mathbb{Z} \, / \,\left\{\left(-1, 0\right),\left(-1, 1\right),\left(0, -1\right),\left(0, 1\right),\left(1, -1\right),\left(1, 0\right)\right\}\right\}$ \\
$\left\{ x_{0}^{2} + {\left(x_{0} + x_{1}\right)} x_{1} \, | \, (x_{0},x_{1})\in \mathbb{Z} \times \mathbb{Z} \, / \,\left\{\left(-1, 0\right),\left(-1, 1\right),\left(0, -1\right),\left(0, 1\right),\left(1, -1\right),\left(1, 0\right)\right\}\right\}$ \\
\end{tabular}$ \\ \hline
$17$ & $\begin{tabular}{l}
$ \left\{ 3 \, x_{0}^{2} \, | \, x_{0}\in \mathbb{Z} \right\} $ \\
$ \left\{ -\frac{3}{4} \, x_{0}^{2} \, | \, x_{0}\in \mathbb{Z} \right\} $ \\
$ \left\{ -3 \, x_{0}^{2} \, | \, x_{0}\in \mathbb{Z} \right\} $ \\
$\left\{ x_{0}^{2} \, | \, x_{0}\in \mathbb{Z} \, / \,\left\{-1,1\right\}\right\}$ \\
$ \left\{ \frac{3}{4} \, x_{0}^{2} \, | \, x_{0}\in \mathbb{Z} \right\} $ \\
$\left\{ -x_{0}^{2} \, | \, x_{0}\in \mathbb{Z} \, / \,\left\{-1,1\right\}\right\}$ \\
\end{tabular}$ \\ \hline
\end{tabular}

\end{scriptsize}

\footnotetext[1]{A positional number of the group in ITA}
\footnotetext[2]{The explicit forms were generated by \url{https://github.com/davendiy/cryst-automata}}

\newpage

\begin{scriptsize}
\begin{table}
\centering
\begin{tabular}{|l|c|c|c|c|c|}
\hline
\textbf{group} & \textbf{minimal alphabet} & \textbf{gens amount} & \textbf{semi-direct} & \textbf{determinant} & \textbf{cond.2} \\
\hline
Group 1 p1    & \textbf{2}     & 2 & + &  &  \\
Group 2 p2    & \textbf{2}     & 3 & + &  &  \\
Group 3 p1m1  & \textbf{4}     & 3 & + & $x_0x_1$ &  \\
Group 4 p1g1  & \textbf{6}     & 2 &   & $x_0x_1$ & $x_1$ odd \\
Group 5 c1m1  & \textbf{4}     & 3 & + & $\left(x_{0} - 2x_{1}\right)x_{0}$ &  \\
Group 6 p2mm  & \textbf{2}     & 4 & + & $-x_0x_1$ &  \\
Group 7 p2mg  & \textbf{6}     & 3 &   & $x_0x_1$ & $x_0$ odd \\
Group 8 p2gg  & \textbf{3}     & 4 &   & $x_0x_1$ & $x_0, x_1$ odd \\
Group 9 c2mm  & \textbf{3}     & 4 & + & $x_{1}(2x_{0} - x_{1})$ &  \\
Group 10 p4   & \textbf{2}     & 4 & + & $x_0^2 + x_1^2$ &  \\
Group 11 p4mm & \textbf{2}     & 5 & + & $-2x_0^2$ &  \\
Group 12 p4gm & \textbf{9}     & 5 &   & $x_0^2$ & $x_0$ odd \\
Group 13 p3   & \textbf{3}     & 3 & + & $x_0^2 + x_0x_1 + x_1^2$ &  \\
Group 14 p3m1 & \textbf{4}     & 4 & + & $x_0^2$ &  \\
Group 15 p31m & \textbf{4}     & 4 & + & $x_0^2$ &  \\
Group 16 p6   & \textbf{3}     & 4 & + & $x_0^2 + x_0x_1 + x_1^2$ &  \\
Group 17 p6mm & \textbf{3}     & 5 & + & $\tfrac{3}{4}x_0^2$ & $x_0$ is even \\
\hline
\end{tabular}
\caption{Summary of groups and their algebraic properties (without binary alphabet notes)}
\label{tab:groups}
\end{table}
\end{scriptsize}


\begin{thebibliography}{20}
  \bibitem{Nekrashevych:Self-similar} V. Nekrashevych. {\it Self-similar groups},
{Mathematical Surveys and Monographs}, Vol.117 (American Mathematical Society, Providence, 2005).

  \bibitem{Szcz:CrystBook} Andrzej Szczepanski. Geometry of crystallographic groups, volume 4
of Algebra and Discrete Mathematics. World Scientific, 2012.
  
\end{thebibliography}

\end{document}
