\documentclass[12pt]{article}
\usepackage{a4wide}
\usepackage{setspace}
\usepackage{amsmath,amssymb,amsthm}
\usepackage{graphicx}

\usepackage[english]{babel}

\usepackage{hyperref}
\hypersetup{
 colorlinks=true,
 linkcolor=blue,
 filecolor=magenta,
 urlcolor=cyan,
 pdftitle={Overleaf Example},
 pdfpagemode=FullScreen,
}

\DeclareMathOperator{\tr}{tr}
\DeclareMathOperator{\Dom}{Dom}
\newtheorem{theorem}{Theorem}
\newtheorem{lemma}{Lemma}

\newtheorem{proposition}[theorem]{Proposition}
\newtheorem{corollary}{Corollary}[theorem]
\newtheorem{conjecture}{Conjecture}

\theoremstyle{definition}
\newtheorem{definition}{Definition}
\newtheorem{example}{Example}
\newtheorem{problem}{Problem}
\newtheorem{question}{Question}
\newtheorem{remark}{Remark}
\newtheorem{algorithm}{Scheme}

% \graphicspath{ {./assets} }


\title{Self-replicating degrees of plane and space groups}
\author{ Davyd Zashkolnyi }


\begin{document}
\maketitle
\begin{abstract}
  A group $G$ admits self-replicating action of degree $d$ if it acts on a space of words over alphabet of size $d$.
  A group $G$ admits .... if and only if it has a subgroup $H$ of index $d$ and a homomorphism
  $\phi : H \rightarrow G$ with trivial core. Every crystallographic group admits a self-replicating action.
  We determine self-replicating degrees for crystallographic groups of dimension $n=2,3$.
\end{abstract}

\tableofcontents

\newpage
\section{Self-similar actions}

\textbf{..........motivation...........}

Let $X$ be a finite set (alphabet) and $X^{*}$ the set of all words over $X$, including the empty word $\epsilon$.
The set $X^{*}$ with operation concatenation of words is a free monoid generated by $X$.
The length of a word $v$ is denoted by $|v|$. The set $X^{*}$ can be also viewed as a $d$-regular rooted tree with $\epsilon$
being a root and $d=|X|$ being the size of the alphabet.

\begin{definition}
  A faithful action of a group $G$ on the set $X^{*}$ is called \textit{self-similar} if for every $g\in G$
  and $x\in X$ there exist $y\in X$ and $h\in G$ such that $g(xw)=yh(w)$ for all $w\in X^{*}$. A group $G$ is
  called \textit{self-similar} if it admits a self-similar action over some alphabet $X$.
\end{definition}

\begin{definition}
  A self-similar action is called \textit{self-replicating} or \textit{recurrent} if for every triple
  $g \in G, x \in X$ and $h \in G$ there exist $y \in X$ such that $g(xw) = yh(w)$.
\end{definition}

Informally, a group is self-similar if its action repeats itself on some subset. Any element $g \in G$ can be defined
accordingly to it's action on $X^*$. Denote $g = \pi (g_1 g_2 \dots g_d)$ for $\pi \in Aut(X)$ and $g_i \in G$. The
element $g$ acts on a word $w \in X^*$ according to its prefix $x_i \in X$:
$$g(x_i w) = \pi(x_i) g_i(w).$$
Here the element $g_i$ is called a \textit{restriction} and is denoted as $g|_{x} = g_i$ for $x = x_i$.
We can also define right multiplication rule $(f * g)(x) = f(g(x))$:
\[a = \pi (a_1, a_2, \dots a_n), \quad b = \sigma (b_1, b_2, \dots b_n)\]
\[a * b = \pi * \sigma (a_{\sigma(1)}*b,\, a_{\sigma(2)}*b,\, \dots \,
a_{\sigma(n)}*b)\]

To study self-similar groups, a convenient toolkit was invented by Nekrashevych in \cite{Nekrashevych:Self-similar},
that is, virtual endomorphisms.

\begin{definition}
  A \textit{virtual endomorphism} $\phi$ of a group $G$ is a homomorphism from its subgroup of
  finite index $\Dom \phi := H < G$.

  A virtual endomorphism $\phi$ is called \textit{simple} if $G$ has no $\phi$-invariant normal subgroups.
\end{definition}

\begin{theorem}(Nekrashevych)
  Every self-replicating action induces a surjective associative virtual endomorphism $\phi_x (g) = g|_x$ for any $x \in X$.
  The index of the domain $[G : \Dom \phi_x]$ equals to size of the alphabet $d=|X|$.

  Every simple surjective virtual endomorphism $\phi : H \rightarrow G$ forms a faithful self-replicating
  action over an alphabet $X = G/H = \{x_i\}$:
  \[
  g \cdot x_i = x_j \cdot \phi(g_j^{-1}gg_i),
  \]
  where $x_i = g_i$ are the right coset representatives of $H$ in $G$, and $j$ is chosen in a way that $g_j^{-1}g g_i \in H$.
\end{theorem}

Virtual endomorphisms were used to analyze self-similar actions for a wide range of groups: free abelian groups,
finitely generated nilpotent groups, solvable groups, wreath products of abelian groups ....

\begin{theorem}(Nekrashevych)\label{th:Nekrashevych-abelian}
  Given a finitely generated free abelian group $G$ and its subgroup of finite index $H$. Then, any surjective virtual
  endomorphism $\phi : H \rightarrow G$ is induced by a linear operator $\phi(g) = Ag$ and is simple if and only if the
  characteristic polynomial of $A$ isn't divisible by a monic polynomial with integer coefficients.
\end{theorem}

\begin{proof}
  ...
\end{proof}


\begin{theorem}
  Every finitely generated virtually abelian group admits a self-similar action. In particular, every crystallographic
  group admits a self-similar action.
\end{theorem}

\begin{proof}
  A finitely generated virtually abelian group $G$ contains $H = \mathbb{Z}^n$ for some $n\in\mathbb{N}$ as a subgroup of
  finite index, which admits a self-similar action by Theorem \ref{th:Nekrashevych-abelian}.

  Let $(H,X^{*})$ be a self-similar action. Let $D$ be a set of coset representatives for $H$ in $G$ with $e\in D$.
  We construct a self-similar action of $G$ over the alphabet $Y=D\times X$. The action is defined by
  the rule: for $g\in G$ and $(d,x)\in Y$,
  \[
  g(d,x)=(c,y) \quad \mbox{ and } \quad g|_{(d,x)}=h|_x,
  \]
  where $c\in D$ is the unique element such that $h=c^{-1}gd\in H$ and $y=h(x)$.
  It is straightforward to check that the action is well-defined. This action is faithful,
  because every $g\in G\setminus H$ acts nontrivially on $Y$ and $H$ acts faithfully on the subspace $(\{e\}\times X)^{*}$ of $Y^{*}$.

  \begin{align*} % the action is well-defined
    g_1g_2 * (d,x) &= g_1*(c,h_2(x))* h_2|_x \qquad && h_2=c^{-1}g_2d\\
    &= (b,h_1(h_2(x))) *h_1|_{h_2(x)}h_2|_x \qquad && h_1=b^{-1}g_1c\\
    (g_1g_2) *(d,x) &= (b,h(x))*h|_x \qquad && h=b^{-1}g_1g_2d=h_1h_2
  \end{align*}

\end{proof}

Despite every virtual abelian group is self-similar, not every one admits a self-replicating action.
As an example, consider $G = \mathbb{Z}_2 \times \mathbb{Z}$ which is not
self-replicating due to \cite{Sidki}. We will prove in the section \ref{sec:cryst} that some of them, particularly
crystallographic groups, are self-replicating. In the section \ref{sec:computations} we will investigate what
alphabets -- meaning size of an alphabet -- appear in self-replicating actions of crystallographic groups in dimensions
2 and 3.


% =====================================================================================================================


\section{Crystallographic groups}\label{sec:cryst}

Crystallographic groups describe symmetries of repeated patterns of an $n$-dimensional space. They were known for
several centuries in case of $n=2$, though the complete list was only given in 1891 by Fedorov. In general, they appeared
as a part of 18th Hilbert's problem, and were characterized by Bieberbach and Zassenhaus later on. There are 17 distinct
groups for $n=2$ up to isometry, known as wallpaper groups or plane groups. Likewise, in $n=3$ there are 219 crystallographic groups,
known as space groups. If, however, space groups are considered up to conjugacy with respect
to orientation-preserving affine transformations, their number is 230.

Crystallographic groups for dimensions $n=2,3$  are thoroughly studied due to their high value in physics. Most of the
symmetric properties and applied results for every group can be found in the International Tables for Crystallography \cite{CrystTables}.
We, however, consider crystallographic groups from the group theoretical perspective, which can be found in \cite{Szcz:CrystBook}.

Let $\mathbf{A}(n)$ be a group of affine transformations and $\mathbf{E}(n)$ a group of isometries of
$\mathbb{R}^n$ respectively.

\begin{definition}
  A \textit{crystallographic group} is a discrete cocompact subgroup of $\mathbf{E}(n)$.
\end{definition}

Recall that $\mathbf{E}(n) < \mathbf{A}(n) = GL_n(\mathbb{R}) \ltimes \mathbb{R}^n$. From this follows that elements of any
crystallographic group can be represented as pairs $(A, t)$ where $A \in GL_n(\mathbb{R})$ and $t \in \mathbb{R}^n$.
Recall also the multiplication rule for a semi-direct product: $(A, a)(B, b) = (AB, Ab + a)$.
We require the following theorems to be able to work with arbitrary crystallographic group.

\begin{theorem}[Bieberbach, 1910]
\begin{enumerate}
  \item If $G \subset \mathbf{E}(n)$ is a crystallographic group, then the set of translations
  $G \cap (I \times \mathbb{R}^n)$ is a torsion free and finitely generated abelian group of rank $n$, and
  is a maximal abelian and normal subgroup of finite index.

  \item Two crystallographic groups of dimension $n$ are isomorphic if and only if they are conjugate
  in the affine group $\mathbf{A}(n)$.

  \item For any $n\in\mathbb{N}$ there are only finite number of isomorphism classes of crystallographic groups of dimension $n$.
\end{enumerate}
\end{theorem}

\begin{theorem}[Zassenhaus]
  A group $G$ is isomorphic to a crystallographic group of dimension $n$ if and only if $G$ has a normal,
  free abelian subgroup $\mathbb{Z}^n$ of finite index, which is a maximal abelian subgroup of $G$.
\end{theorem}

In other words, a crystallographic group $G$ is some finitely generated virtually abelian subgroup of affine group $\mathbf{A}(n)$.
It can be uniquely defined via a triplet $(P, L, \alpha)$ of a \textit{crystallographic point group}
$P < GL_n(\mathbb{R})$, a $P$-invariant \textit{lattice} or \textit{translation group} $L \subset \mathbb{R}^n$ and a
map $\alpha : P \rightarrow \mathbb{R}^n$. The former two can be ordered in a short exact sequence
$L \rightarrow G \rightarrow P$ and the entire group is expressed as $G = \{ (A, \alpha(A) + t) \,|\, A \in P, \, t \in L \}$.

By the Bieberbach theorem, $L$ is the maximal normal free
abelian subgroup of $G$. The image of $\alpha$ is called a \textit{System of Non-trivial
Translations (SNoT)}. Simply, these are elements that make $G$ different from a semi-direct product $P \ltimes L$. It is
also worth noting that the mapping $\Omega : g \mapsto \alpha(g) + L, g \in P$ is a one-dimensional cocycle on $P$ with
values in $\mathbb{R}^n$ / L. Given a crystallographic group $G = (P, L, \alpha)$ denote $\Pi(G) := P$ and $T(G) := L$.
(\textbf{Question: do we need to choose basis already?})

\begin{definition}
  A crystallographic group $G$ which is isomorphic to a semi-direct product $\Pi(G) \ltimes T(G)$ of its point group and lattice is
  called \textit{symmorphic}.
\end{definition}

In this paper we focus mainly on self-replicating actions of abstract crystallographic groups. Recall that a self-replicating
action is induced by a surjective virtual endomorphism. The next theorem gives us a way to connect self-replicating
actions to the results of Bieberbach.

\begin{theorem}(Bondarenko)\label{th:surjection-is-bijection}
  A surjective virtual endomorphism of crystallographic group is an isomorphism from its subgroup to itself.
\end{theorem}

\begin{corollary}
  For every surjective virtual endomorphism $\phi : H \rightarrow G$ of a crystallographic group $G$ there exists an
  affine element $(A, t) \in \mathbf{A}(n)$ such that $\phi(g) = (A, t)^{-1}g(A, t)$ and $H = \phi^{-1}(G) = (A, t)G(A, t)^{-1}$.
\end{corollary}

Let us investigate how surjective virtual endomorphisms act on crystallographic groups, deriving from
the theorem \ref{th:surjection-is-bijection}.
Consider a crystallographic group $G$, its isomorphic subgroup of finite index $H < G$
and a virtual endomorphism $\phi : H \rightarrow G$ which is given by $\phi(g) = (A, t)^{-1}g(A, t)$.
Its action on the generators of $T(G)$ is reduced to a linear operator:

\begin{equation}
  (A, t)(E, e_i)(A, t)^{-1} = (E, Ae_i + t - AA^{-1}t) = (E, At). \label{eq:trans_gens}
\end{equation}

\begin{lemma}\label{lemma:equal_index}
  Let $G$ be a crystallographic group and $H<G$ be a subgroup isomorphic to $G$.
  Then $T(H)=T(G)\cap H$ and $[G:H] = [T(G):T(H)]$.
\end{lemma}
\begin{proof}
  The equality $T(H)=H\cap T(G)$ holds by the definition of the translation subgroup. Since $G$ and $H$ are isomorphic,
  they have the same point groups. Then the subgroup $HT(G)$ contains all translations and linear parts of $G$, and
  therefore $HT(G)=G$. We can apply the second isomorphism theorem to the subgroups $H$ and $T(G)$, and get $[H:T(H)]=[G:T(G)]$.
  The index of $T(H)$ in $G$ can be expressed in two ways as
  \[
  [G:T(H)]=[G:H][H:T(H)]=[G:T(G)][T(G):T(H)].
  \]
  The equality $[G:H] = [T(G):T(H)]$ follows.
\end{proof}

From the lemma \ref{lemma:equal_index} and the equation (\ref{eq:trans_gens}) follows that $T(H) = AT(G) < T(G)$ and its index
equals to the $\det(A)$. We can now choose the basis equal to the generators of $T(G)$. Doing so, we get that the
lattice $T(G)$ equals to $\mathbb{Z}^n$,
the point group $\Pi(G)$ consists of invertible integral matrices, i.e. $\Pi(G) < GL_n(\mathbb{Z})$, SNoT $\alpha(G)$
consists of vectors from $[0, 1]^n$ and the matrix
$A \in GL_n(\mathbb{Q})$ has integral entities. Note that $t$ can be arbitrary vector from $\mathbb{Q}^n$. More detailed
analysis of admissible pairs $(A, t)$ will be given in the section \ref{sec:computations}.

\begin{theorem}
  A surjective virtual endomorpism $\phi : H \rightarrow G$ given by $\phi(g) = (A, t)^{-1}g(A, t)$ is simple if and
   only if characteristic polynomial of $A^{-1}$ isn't divisible by a monic polynomial with integral coefficients.
\end{theorem}
\begin{proof}

  Denote $L = T(H)$. We will prove that $\phi$ is simple if and only if its restriction on the tranlsation group
  $\phi|_{L} : L \rightarrow G$ is simple as virtual endomorphism of $G$. The statement of the theorem follows from this
  combined with Theorem \ref{th:Nekrashevych-abelian}.

  The $\phi|_{L}$-core is a subgroup of the $\phi$-core. Therefore, $\phi$ being simple implies $\phi|_{L}$ is simple.

  Conversely, assume that the $\phi$-core $K$ is nontrivial. Since crystallographic groups do not contain finite normal subgroups,
  $K$ is infinite and contains translations. For every $m\in\mathbb{N}$, the subgroup $K_m=\langle a^{m}, a\in K\rangle$
  is nontrivial, normal in $G$, and $\phi$-invariant. Taking $m = lcm(order(B) \, | \, B \in P)$ we have $a^{m}\in L$
  for every $a\in G$. Hence, $K_m\leq L$ and belongs to the $\phi|_{L}$-core. And it is not empty since $K$ contains translations.

\end{proof}

We can now make an upper bound for self-replicating degrees of crystallographic groups:

\begin{theorem}
  Every crystallographic group $G = (P, \mathbb{Z}^n, \alpha)$ admits a self-replicating action over an alphabet
  of size $(m+1)^n$ where $m$ is the least common multiplier of denominators of entities of $\alpha(B)$
  for every $B \in P$.
\end{theorem}

\begin{proof}
	Since $A$ is scalar it commutes with any matrix and thus lies in the normalizer of $P$.
	Consider $A = (m+1)E_n$ where $m = lcm \left(m_i : (B, \frac{p}{m_i}) \in G \right)$ is a least common
  multiplier of denominators of translations from the SNoT. Then $A^{-1}$ has one eigenvalue $\lambda = \frac{1}{m}$ and
  the only thing left to check is whether $(A, 0)G (A, 0)^{-1} \subset G$. Using multiplication rules:
	$$
	\phi^{-1}((B, t)) = (A, 0) (B, t) (A, 0)^{-1} = (B, At) = (B, (m+1)t) = (B, t) + (E, mt)
	$$
	where $mt \in \mathbb{Z}^n$ due to the choice of $m$. As a result, $\phi^{-1}(G) < G$.
\end{proof}

\begin{corollary}
  Every planar group admits a self-replicating action over an alphabet of size $\le 9$.
\end{corollary}

\begin{corollary}
  Every symmorphic crystallographic group admits a self-replicating action over an alphabet of size $\le 2^n$.
\end{corollary}


% =====================================================================================================================


\section{Computing self-replicating degree of crystallographic groups}\label{sec:computations}


Let $G$ be a crystallographic group. We consider two sets of positive integers associated with $G$:
\begin{align*}
SCD(G) &= \{d \in \mathbb{N} \, | \, \text{ $G$ has a subgroup $H$ of index $d$ isomorphic to $G$ } \},\\
SRD(G) &= \{d \in \mathbb{N}\, | \, \text{ $G$ admits a self-replicating action of degree $d$ } \}.
\end{align*}
It is clear that $SRD\subset SCD$.
We denote by $MinSRD(G)$ the minimal degree of a self-replicating action of $G$.

From a computational perspective, both $SCD(G)$ and $SRD(G)$ can be viewed as decision problems for the group $G$: given an integer $d$, determine whether $G$ contains a subgroup $H \le G$ of index $d$ with $H \cong G$, or whether $G$ admits a self-replicating action of degree $d$.

\vspace{0.2cm}
\textbf{Computing self-covering degrees.}
Let us describe a method for computing the set $SCD(G)$ for a given crystallographic group $G=(P,\mathbb{Z}^n,\alpha)$. A positive integer $d$ belongs to $SCD(G)$ if and only if there exists an affine transformation $a=(A,t)\in Aff(\mathbb{Q})$ such that $d=|\det A|$ and $a G a^{-1}\leq G$. This condition splits into three parts:
\begin{enumerate}
  \item[1)] Point group compatibility: $APA^{-1}=P$.
  \item[2)] Lattice compatibility: $A\mathbb{Z}^n\leq \mathbb{Z}^n$, which means $A$ must have integer entries.
  \item[3)] Cocycle compatibility: for every $B\in P$,
\begin{equation}\label{eqn:cocycle_compatibility}
\alpha(ABA^{-1})\equiv(E-ABA^{-1})t+A\alpha(B) \ (\bmod\ \mathbb{Z}^n).
\end{equation}
\end{enumerate}
To determine possible values of $|\det A|$, we first describe the integer matrices $A$ satisfying the first condition and then identify those for which there exists a translation $t$ satisfying the third condition.

The integer matrices $A$ satisfying the first condition can be described as follows.
Let $\sigma \in \mathrm{Aut}(P)$ be an automorphism of the point group.
Consider the system of linear equations over the integers
\[
X B = \sigma(B) X \quad \mbox{ for all $B \in P$},
\]
where $X \in \mathrm{Mat}_{n \times n}(\mathbb{Z})$ is the unknown matrix. The set $\mathcal{X}_\sigma$ of all solutions  is a finitely generated $\mathbb{Z}$-module, and there exists a finite, computable set of integral matrices $X_1, \dots, X_m \in \mathcal{X}_\sigma$ such that every $X \in \mathcal{X}_\sigma$ can be written as
\begin{equation}\label{eqn:X_sum_Xi}
X = \sum_{i=1}^{m} x_i X_i, \quad x_i \in \mathbb{Z}.
\end{equation}
The total set $\mathcal{X}$ of matrices $A$ satisfying 1) is the union of $\mathcal{X}_\sigma$ over all $\sigma \in \mathrm{Aut}(P)$.

Next, we determine those matrices $X\in \mathcal{X}$ for which a translation $t$ exists satisfying the cocycle compatibility condition. Let $\sigma \in \mathrm{Aut}(P)$ and $X\in \mathcal{X}_\sigma$ be written as the linear combination (\ref{eqn:X_sum_Xi}) with unknown $x_i\in\mathbb{Z}$. The equation (\ref{eqn:cocycle_compatibility}) for the matrix $X$ can be written as a system of congruence: for every $B\in P$,
\begin{equation}\label{eqn:cocycle_condition}
\sum_{i=1}^{m} x_i X_i \alpha(B) - \alpha(\sigma(B)) \equiv (\sigma(B) - E)t \quad (\bmod\ \mathbb{Z}^n).
\end{equation}
in the unknown $x_i\in\mathbb{Z}$ and $t\in\mathbb{Q}^n$.
The congruence system is solvable in $t\in\mathbb{Q}^n$ if and only if, for every $B\in P$ and $v\in\mathbb{Q}^n$ such that $v^{\top}(\sigma(B)-E)\in\mathbb{Z}^n$,
\[
\sum_{i=1}^m x_i(v^{\top} X_i\alpha(B))\equiv v^{\top}\alpha(\sigma(B)) \quad (\bmod \ 1).
\]
This yields a finite system of linear congruences in the integer variables
$x_i$. Solving this system in general form and substituting the solution into $X$, we obtain a finite, computable set of integral matrices $Y_0, Y_1, \dots, Y_r$. The subset $\mathcal{Y}_\sigma\subseteq \mathcal{X}_\sigma$ consisting of all matrices satisfying conditions 1)--3) is then given by
\begin{equation}\label{eqn:Y_sum_Yi}
Y = Y_0+\sum_{i=1}^{r} y_i Y_i, \quad y_i \in \mathbb{Z}.
\end{equation}
The determinant of $Y$ can be expressed as an integer polynomial $f_\sigma(y_1,\ldots,y_r)$.
Let $\mathcal{Y}$ be the union of $\mathcal{Y}_\sigma$ over $\sigma \in \mathrm{Aut}(P)$.

Finally, we get the following description of the set of self-covering and self-replicating degrees:
\begin{align*}
SCD(G)&=\{ |\det A|\neq 0 : A\in \mathcal{Y}\}=\{ |f_\sigma(y_1,\ldots, y_r)|\neq 0 : y_i\in\mathbb{Z} \mbox{ and } \sigma\in Aut(P) \},\\
SRD(G)&=\{ |\det A|\neq 0 : A\in \mathcal{Y} \mbox{ and no eigenvalue of $A^{-1}$ is an algebraic integer}\}.
\end{align*}
The set of integer values of an integer polynomial generally admits no simple or computable arithmetic description.
Determining whether a given integer $d>0$ belongs to $SCD(G)$ amounts to deciding the solvability of the Diophantine equation $|f_\sigma(y_1,\ldots, y_r)|=d$ for some $\sigma\in Aut(P)$. By Matiyasevich's theorem, such problems are undecidable for arbitrary polynomials. We do not know whether the problem is decidable in our settings:

\begin{problem}
Let $G$ be a crystallographic group. 

(1) Is it decidable, for a given integer $d$, whether there exists a subgroup $H<G$ of index $d$ with $H\cong G$? Equivalently, does the orbifold $\mathbb{R}^n/G$ admit a self-covering of degree $d$?

(2) Is it decidable, for a given integer $d$, whether the group $G$ admits a self-replicating action of degree $d$?
\end{problem} 

Next we show how to describe the sets $SCD(G)$ and $SRD(G)$ for crystallographic groups in dimension two. In this case, every determinant polynomial $f_\sigma$ is a quadratic form, and one can explicitly describe the set $SCD(G)$. The case of $SRD(G)$ is more subtle; we reduce the condition on the eigenvalues to a simpler arithmetic condition.

\begin{proposition}
Let $A\in M_2(\mathbb{Z})$ and $\det(A)\neq 0$. No eigenvalue of $A^{-1}$ is an algebraic integer if and only if $\det(A)\neq \pm 1$ and $\det(A) \pm \operatorname{tr}(A) \neq -1$.
\end{proposition}
\begin{proof}
The characteristic polynomials of $A$ and $A^{-1}$ are
\begin{align*}
\chi_A(x)&=x^2-\textrm{tr}(A) x + \det(A)\in\mathbb{Z}[x],\\
\chi_{A^{-1}}(x)&=x^2-\frac{\textrm{tr}(A)}{\det(A)} x + \frac{1}{\det(A)}\in\mathbb{Q}[x].
\end{align*}
If $\det(A)=\pm 1$, then $\chi_{A^{-1}}(x)\in\mathbb{Z}[x]$ and its roots are algebraic integers. If $\det(A) \pm \textrm{tr}(A) = -1$, then $A^{-1}$ has an eigenvalue $1$ or $-1$. 

Conversely, assume $\chi_{A^{-1}}(x)$ is divisible by a monic polynomial $p(x)\in\mathbb{Z}[x]$. If $\deg p(x)=2$, then $p(x)=\chi_{A^{-1}}(x)$ and $\det(A)=\pm 1$. If $\deg p(x)=1$ and $p(x)=x-a$, then $a$ is an integer eigenvalue of $A^{-1}$. Now $a^{-1}$ is a root of a monic integer polynomial $\chi_A(x)$. Since $a$ and $a^{-1}$ are both algebraic integer, we have $a=\pm 1$ and
\[
\chi_{A}(\pm 1)=1\mp \textrm{tr}(A)+\det(A)=0 \quad \Rightarrow  \quad \det(A)\mp \textrm{tr}(A)=-1.
\]
\end{proof}

The trace of a matrix $Y\in \mathcal{Y}_\sigma$ written in the form (\ref{eqn:Y_sum_Yi}) can be expressed by the polynomial $g_\sigma(y_1,\ldots,y_r)$. This leads us to the following characterization of $SRD(G)$:
\begin{align*}
SRD(G) = \big\{ | f_\sigma(y_1,\ldots, y_r)|>1 \ : & \ \sigma\in Aut(P) \text{ and } y_i\in\mathbb{Z} \text{ such that } \\
& \quad ( f_\sigma \pm g_\sigma)(y_1,\ldots,y_r)\neq -1 \big\}.
\end{align*}

Similar approach is possible for dimension three. However, the point groups of three-dimensional crystallographic groups have a particularly nice form, which allows to describe $SRD$ in terms of $SRD$ for two-dimensional case.

\begin{proposition}
Point groups in three dimension.
\end{proposition}







\vspace{2cm}
\begin{proposition}
  Let $A_\sigma$ be a general solution of the linear system $XB = \sigma(B)X, B \in P$, written in terms of free variables
  $x_1, x_2, \dots x_m, m \le n^2$ and their linear combinations. Then $\det(A_\sigma)$ is a homogeneous polynomial of degree $n$
  over $x_i$. (\textbf{Question: why $m$ is always less than $n$? Does it come from $B \in P$ being integral of $\det B=1$?})
\end{proposition}

\begin{proof}
  Let $X$ be a matrix of $n^2$ indeterminates $x_i$. Then the system $XB - \sigma(B)X = 0$ is a homogeneous system
  of $d * n^2$ linear equations for $d = |P|$. If the system is not singular, then it has infinite amount of solutions for $m \le n$
  independent variables and their linear combinations. By definition, the determinant equals to
  $$\det A = \sum_{\pi} sign(\pi) \prod_{i} a_{i\pi(i)}$$
  which is a sum of multiples of some $n$ elements of $A$. Since every element of $A$ is either 0 or a linear combination
  of free variables, then $\det A$ is either homogeneous polynomial of degree $n$ or 0.
\end{proof}



Dimension two.

Dimension three.

%The first condition ensures that the linear part of the affine conjugation preserves the point group, while the second expresses the compatibility of the translational part with the group cocycle $\alpha$.

\vspace{2cm}
\begin{enumerate}
 \item ($SRDegrees$) solve the equations
 \begin{equation}\label{eq:det}
 |\det{A}| \neq 1
 \end{equation}
 and
 \begin{equation}\label{eq:eigenvalues}
 \det{A} \pm \tr{A} \neq -1
 \end{equation}
 to exclude combinations that form virtual endomorphisms that aren't simple.

 \begin{theorem}
  The equations (\ref{eq:det}) and (\ref{eq:eigenvalues}) can be solved algorithmically for dim=2.
 \end{theorem}
 \begin{proof}
 \end{proof}

\end{enumerate}




Consider the planar case $G = (P, \mathbb{Z}^2, \alpha)$
Let $(A, t)$ be an affine element which induces surjective virtual endomorphism
$\phi : H \rightarrow G, \, \phi(g) = (A, t)^{-1}g(A, t)$, where $H = (A, t) G (A, t)^{-1} \simeq G$.
From surjectivity follows that $A$ has integral entities and $|\det(A)| > 1$. Denote
$$A = \left(\begin{array}{rr}
x_{0} & x_{1} \\
x_{2} & x_{3}
\end{array}\right).$$


\begin{proposition}
A surjective virtual endomorphism, generated by $(A, t)$, is simple if and only if
$\det{A} \pm \tr{A} \neq -1$.
\end{proposition}

\begin{proof}
$A^{-1}$ is simple if and only if its characteristic polynomial $f(x)$ isn't divisible by a monic polynomial with
integral coefficients. For $\dim =2$ it means that $f(x)$ has no integral coefficient and no integral root. $A$ has an
eigenvalue $\lambda$ if and only if $1/\lambda$ is an eigenvalue of $A^{-1}$. Let $\lambda_1, \lambda_2$ be two
eigenvalues of $A$. We can write an explicit form of $f(x)$:
$$f(x) = x^2 - \left(\frac{1}{\lambda_1} + \frac{1}{\lambda_2}\right)x + \dfrac{1}{\lambda_1 \lambda_2} =
x^2 - \dfrac{\lambda_1 + \lambda_2}{\lambda_1 \lambda_2}x + \frac{1}{\lambda_1 \lambda_2} =
x^2 - \dfrac{\lambda_1 + \lambda_2}{\det(A)}x + \frac{1}{\det(A)}$$

From this follows, that $f(x)$ have at least one rational coefficient for $|\det(A)| >1$.
Let's now consider $h(x)$ the characteristic polynomial of $A$. It's explicit form:
$$h(x) = x^2 - (\lambda_1 + \lambda_2)x + \lambda_1\lambda_2 = x^2 + dx + \det(A).$$
$A$ has integral entities, therefore $h(x)$ has integral coefficients as well. By rational root theorem, $p/q$
is a root of integral polynomial if and only if $q$ is an integer factor of the leading coefficient. $h(x)$ is monic
as a characteristic polynomial. As a result, every rational eigenvalue of $A$ is integral, so the only possible
integral eigenvalue of $A^{-1}$ is $\pm1$.

Finally, let's write conditions on the coefficients:
$$h(\pm1) = 1  \mp (x_0 + x_3) + (x_0x_3 - x_1x_2) \neq 0$$
$$\det(A) \pm(x_0 + x_3) \neq -1$$
\end{proof}


\newpage
\subsection{ Group 7 p2mg }


The Group 7 ($p2mg$) from ITA is generated by the following elements:
$$G_7 = \left\langle\left(\begin{array}{rr|r}
1 & 0 & 1 \\
0 & 1 & 0 \\
\end{array}\right),\left(\begin{array}{rr|r}
-1 & 0 & 1/2 \\
0 & 1 & 0 \\
\end{array}\right)\left(\begin{array}{rr|r}
-1 & 0 & 0 \\
0 & -1 & 0 \\
\end{array}\right)  \right\rangle = \langle a,b,c \rangle$$
The normalizer of the point group consists of matrices of two types:
$$A_1, A_2 = \left(\begin{array}{rr}
0 & x_{1} \\
x_{0} & 0
\end{array}\right), \left(\begin{array}{rr}
x_{0} & 0 \\
0 & x_{1}
\end{array} \right) \quad   \,  x_0,x_1 \in \mathbb{Q}^*$$

Solving the system of congruences from 2) we obtain that for $A_1$ system doesn't have solutions.

For the $A_2$ we get that $x_0$ should be odd integral, and $x_1$ integral.

All the conjugations are generated by
$$
(A, t) = \left(\begin{array}{rr|r}
x_0 & 0 & a_0 \\
0 & x_1 & a_1 \\
\hline
0 & 0 & 1
\end{array}\right)
$$
where $x_0 = 2k-1, x_1 \in Z, a_0 a_1 \in 1/2 Z$.

Therefore, $$SCDegrees = \{ |\det(A_2)| = |x_0 x_1|, \, |\,  x_0 \in 2\mathbb{Z} + 1, \, x_1 \in \mathbb{Z}\} = \mathbb{N}$$
The inverse of matrix $A_2$ fits the condition 3) if and only if $|x_0| > 1, |x_1| > 1$. Therefore,

$$SRDegrees = \{ (2k + 1)(m+1) \, | \, k,m \in \mathbb{N}\}$$
or not primes and not powers of 2. In particular, the minimal self-replicating degree is 6 when $x_0 = 3, x_1 = 2$.

\subsection{Group 13}

The Group 13 (p3) from ITA is generated by the following elements:
$$G_{13} = \left\langle\left(\begin{array}{rrr}
0 & -1 & 0 \\
1 & -1 & 0 \\
0 & 0 & 1
\end{array}\right), \left(\begin{array}{rrr}
1 & 0 & 1 \\
0 & 1 & 0 \\
0 & 0 & 1
\end{array}\right), \left(\begin{array}{rrr}
1 & 0 & 0 \\
0 & 1 & 1 \\
0 & 0 & 1
\end{array}\right)\right\rangle$$
The normalizer of the point group consists of matrices of two types:
$$A_1, A_2 = \left(\begin{array}{rr}
-x_{0} & x_{0} + x_{1} \\
x_{1} & x_{0}
\end{array}\right), \left(\begin{array}{rr}
x_{0} + x_{1} & -x_{0} \\
x_{0} & x_{1}
\end{array}\right), \quad x_0,x_1 \in \mathbb{Q}^*$$
The group $G_{13}$ is symmorphic, i.e. $G_{13} = P \ltimes \mathbb{Z}^2$, from which follows that congruences from 2)
are trivial and the only condition for translation is to be integral.

$$SCDegrees = \{ |\det(A_1)|, |\det(A_2)|\} = \{ |x_{0}^{2} + {\left(x_{0} + x_{1}\right)}| \quad | \quad x_0, x_1 \in \mathbb{Z}\}$$

\subsubsection{Self-replicating degrees.}
\begin{enumerate}
\item Consider the first matrix
$$A = \left(\begin{array}{rr}
x_{0} + x_{1} & -x_{0} \\
x_{0} & x_{1}
\end{array}\right)$$

For the determinanat we have $\det{A} = x_{0}^{2} + {\left(x_{0} + x_{1}\right)} x_{1} \in \mathbb{Z}$.
The simplicity of the respective virtual endomorphism depends on the following diophantine equations:
$$x_{0}^{2} + {\left(x_{0} + x_{1}\right)} x_{1} \neq \pm 1$$
$$x_{0}^{2} + {\left(x_{0} + 2\right)} x_{1} + x_{1}^{2} + x_{0} \neq -1$$
$$x_{0}^{2} + {\left(x_{0} - 2\right)} x_{1} + x_{1}^{2} - x_{0} \neq -1$$

Solving the first equation on the determinant $\det{A} \neq \pm 1$ we obtain the pairs:
$$\left[\left(0, 1\right), \left(-1, 1\right), \left(1, -1\right), \left(-1, 0\right), \left(1, 0\right), \left(0, -1\right)\right]$$

Solving next equations on the eigenvalues $\det{A} \pm \tr{A} = -1$:
$$\left[\left(0, -1\right), \left(0, 1\right)\right]$$

\item Consider the second matrix
$$A = \left(\begin{array}{rr}
-x_{1} & x_{0} + x_{1} \\
x_{0} & x_{1}
\end{array}\right)$$
Determinant: $\det{A} = -{\left(x_{0} + x_{1}\right)} x_{0} - x_{1}^{2} \in \mathbb{Z}$.
Self-replicating degrees:
$$-{\left(x_{0} + x_{1}\right)} x_{0} - x_{1}^{2} \neq \pm 1$$
$$-x_{0}^{2} - x_{0} x_{1} - x_{1}^{2} \neq -1$$
$$-x_{0}^{2} - x_{0} x_{1} - x_{1}^{2} \neq -1$$
Solving diophantine equation for the determinant $\det{A} = \pm 1$:
$$\left[\left(0, 1\right), \left(-1, 1\right), \left(1, -1\right), \left(-1, 0\right), \left(1, 0\right), \left(0, -1\right)\right]$$
Solve diophantine equation for eigenvales:
$$\left[\left(0, 1\right), \left(-1, 1\right), \left(1, -1\right), \left(-1, 0\right), \left(1, 0\right), \left(0, -1\right)\right]$$
$$\left[\left(0, 1\right), \left(-1, 1\right), \left(1, -1\right), \left(-1, 0\right), \left(1, 0\right), \left(0, -1\right)\right]$$
\end{enumerate}


\newpage
\begin{scriptsize}
\begin{tabular}{|l|l|l|l|} \hline

matrix & det & eigenvalue != 1 & eigenvalue != -1 \\ \hline \hline
$\left(\begin{array}{rr}
x_{0} + x_{1} & -x_{0} \\
x_{0} & x_{1}
\end{array}\right)$ & $x_{0}^{2} + {\left(x_{0} + x_{1}\right)} x_{1}$ & $x_{0}^{2} + {\left(x_{0} + 2\right)} x_{1} + x_{1}^{2} + x_{0} + 1$ & $x_{0}^{2} + {\left(x_{0} - 2\right)} x_{1} + x_{1}^{2} - x_{0} + 1$ \\ \hline
$\left(\begin{array}{rr}
-x_{1} & x_{0} + x_{1} \\
x_{0} & x_{1}
\end{array}\right)$ & $-{\left(x_{0} + x_{1}\right)} x_{0} - x_{1}^{2}$ & $-x_{0}^{2} - x_{0} x_{1} - x_{1}^{2} + 1$ & $-x_{0}^{2} - x_{0} x_{1} - x_{1}^{2} + 1$ \\ \hline
\end{tabular}
\\
\begin{tabular}{|l|l|l|} \hline
matrix & det & except \\ \hline \hline
$\left(\begin{array}{rr}
x_{0} + x_{1} & -x_{0} \\
x_{0} & x_{1}
\end{array}\right)$ & $x_{0}^{2} + {\left(x_{0} + x_{1}\right)} x_{1}$ & $\left\{\left(-1, 0\right),\left(-1, 1\right),\left(0, -1\right),\left(0, 1\right),\left(1, -1\right),\left(1, 0\right)\right\}$ \\ \hline
$\left(\begin{array}{rr}
-x_{1} & x_{0} + x_{1} \\
x_{0} & x_{1}
\end{array}\right)$ & $-{\left(x_{0} + x_{1}\right)} x_{0} - x_{1}^{2}$ & $\left\{\left(-1, 0\right),\left(-1, 1\right),\left(0, -1\right),\left(0, 1\right),\left(1, -1\right),\left(1, 0\right)\right\}$ \\ \hline
\end{tabular}

\end{scriptsize}


\newpage
As a result, we run the proposed algorithm for every planar group and get the following self-repicating degrees.
For shorter notation, we use $OZ := \{2k + 1 | k \in \mathbb{Z}\}$ i.e. odd integers.

\begin{scriptsize}

\begin{tabular}{|l|l|} \hline
num\footnotemark[1]{} & srdegrees\footnotemark[2]{} \\ \hline \hline
$3$ & $\begin{tabular}{l}
$\left\{ x_{0} x_{1} \, | \, (x_{0},x_{1})\in \mathbb{Z} \times \mathbb{Z} \, / \,\left\{\left(n_{1}, -1\right),\left(n_{1}, 1\right),\left(-1, -1\right),\left(-1, 1\right),\left(-1, n_{1}\right),\left(1, -1\right),\left(1, 1\right),\left(1, n_{1}\right)\right\}\right\}$ \\
\end{tabular}$ \\ \hline
$4$ & $\begin{tabular}{l}
$\left\{ x_{0} x_{1} \, | \, (x_{0},x_{1})\in OZ \times \mathbb{Z} \, / \,\left\{\left(n_{1}, -1\right),\left(n_{1}, 1\right),\left(-1, -1\right),\left(-1, 1\right),\left(-1, n_{1}\right),\left(1, -1\right),\left(1, 1\right),\left(1, n_{1}\right)\right\}\right\}$ \\
\end{tabular}$ \\ \hline
$5$ & $\begin{tabular}{l}
$\left\{ {\left(x_{0} - 2 \, x_{1}\right)} x_{0} \, | \, (x_{0},x_{1})\in \mathbb{Z} \times \mathbb{Z} \, / \,\left\{\left(-1, -1\right),\left(1, 1\right),\left(2 \, t_{0} + 1, t_{0}\right),\left(-1, 0\right),\left(-1, n_{1}\right),\left(2 \, t_{0} - 1, t_{0}\right),\left(1, 0\right),\left(1, n_{1}\right)\right\}\right\}$ \\
\end{tabular}$ \\ \hline
$6$ & $\begin{tabular}{l}
$\left\{ -x_{0} x_{1} \, | \, (x_{0},x_{1})\in \mathbb{Z} \times \mathbb{Z} \, / \,\left\{\left(-1, -1\right),\left(-1, 1\right),\left(1, -1\right),\left(1, 1\right)\right\}\right\}$ \\
$\left\{ x_{0} x_{1} \, | \, (x_{0},x_{1})\in \mathbb{Z} \times \mathbb{Z} \, / \,\left\{\left(n_{1}, -1\right),\left(n_{1}, 1\right),\left(-1, -1\right),\left(-1, 1\right),\left(-1, n_{1}\right),\left(1, -1\right),\left(1, 1\right),\left(1, n_{1}\right)\right\}\right\}$ \\
\end{tabular}$ \\ \hline
$7$ & $\begin{tabular}{l}
$\left\{ x_{0} x_{1} \, | \, (x_{0},x_{1})\in \mathbb{Z} \times OZ \, / \,\left\{\left(n_{1}, -1\right),\left(n_{1}, 1\right),\left(-1, -1\right),\left(-1, 1\right),\left(-1, n_{1}\right),\left(1, -1\right),\left(1, 1\right),\left(1, n_{1}\right)\right\}\right\}$ \\
\end{tabular}$ \\ \hline
$8$ & $\begin{tabular}{l}
$\left\{ -x_{0} x_{1} \, | \, (x_{0},x_{1})\in OZ \times OZ \, / \,\left\{\left(-1, -1\right),\left(-1, 1\right),\left(1, -1\right),\left(1, 1\right)\right\}\right\}$ \\
$\left\{ x_{0} x_{1} \, | \, (x_{0},x_{1})\in OZ \times OZ \, / \,\left\{\left(n_{1}, -1\right),\left(n_{1}, 1\right),\left(-1, -1\right),\left(-1, 1\right),\left(-1, n_{1}\right),\left(1, -1\right),\left(1, 1\right),\left(1, n_{1}\right)\right\}\right\}$ \\
\end{tabular}$ \\ \hline
$9$ & $\begin{tabular}{l}
$\left\{ -x_{0}^{2} + 2 \, x_{0} x_{1} \, | \, (x_{0},x_{1})\in \mathbb{Z} \times \mathbb{Z} \, / \,\left\{\left(-1, -1\right),\left(-1, 0\right),\left(1, 0\right),\left(1, 1\right)\right\}\right\}$ \\
$\left\{ {\left(x_{0} - 2 \, x_{1}\right)} x_{0} \, | \, (x_{0},x_{1})\in \mathbb{Z} \times \mathbb{Z} \, / \,\left\{\left(-1, -1\right),\left(1, 1\right),\left(2 \, t_{0} + 1, t_{0}\right),\left(-1, 0\right),\left(-1, n_{1}\right),\left(2 \, t_{0} - 1, t_{0}\right),\left(1, 0\right),\left(1, n_{1}\right)\right\}\right\}$ \\
\end{tabular}$ \\ \hline
$10$ & $\begin{tabular}{l}
$\left\{ x_{0}^{2} + x_{1}^{2} \, | \, (x_{0},x_{1})\in \mathbb{Z} \times \mathbb{Z} \, / \,\left\{\left(-1, 0\right),\left(0, -1\right),\left(0, 1\right),\left(1, 0\right)\right\}\right\}$ \\
$\left\{ -x_{0}^{2} - x_{1}^{2} \, | \, (x_{0},x_{1})\in \mathbb{Z} \times \mathbb{Z} \, / \,\left\{\left(-1, 0\right),\left(0, -1\right),\left(0, 1\right),\left(1, 0\right)\right\}\right\}$ \\
\end{tabular}$ \\ \hline
$11$ & $\begin{tabular}{l}
$\left\{ x_{0}^{2} \, | \, x_{0}\in \mathbb{Z} \, / \,\left\{-1,1\right\}\right\}$ \\
$ \left\{ 2 \, x_{0}^{2} \, | \, x_{0}\in \mathbb{Z} \right\} $ \\
$ \left\{ -2 \, x_{0}^{2} \, | \, x_{0}\in \mathbb{Z} \right\} $ \\
$\left\{ -x_{0}^{2} \, | \, x_{0}\in \mathbb{Z} \, / \,\left\{-1,1\right\}\right\}$ \\
\end{tabular}$ \\ \hline
$12$ & $\begin{tabular}{l}
$ \left\{ -2 \, x_{0}^{2} \, | \, x_{0}\in \mathbb{Z} \right\} $ \\
$ \left\{ 2 \, x_{0}^{2} \, | \, x_{0}\in \mathbb{Z} \right\} $ \\
$\left\{ x_{0}^{2} \, | \, x_{0}\in OZ \, / \,\left\{-1,1\right\}\right\}$ \\
$\left\{ -x_{0}^{2} \, | \, x_{0}\in OZ \, / \,\left\{-1,1\right\}\right\}$ \\
\end{tabular}$ \\ \hline
$13$ & $\begin{tabular}{l}
$\left\{ -{\left(x_{0} + x_{1}\right)} x_{0} - x_{1}^{2} \, | \, (x_{0},x_{1})\in \mathbb{Z} \times \mathbb{Z} \, / \,\left\{\left(-1, 0\right),\left(-1, 1\right),\left(0, -1\right),\left(0, 1\right),\left(1, -1\right),\left(1, 0\right)\right\}\right\}$ \\
$\left\{ x_{0}^{2} + {\left(x_{0} + x_{1}\right)} x_{1} \, | \, (x_{0},x_{1})\in \mathbb{Z} \times \mathbb{Z} \, / \,\left\{\left(-1, 0\right),\left(-1, 1\right),\left(0, -1\right),\left(0, 1\right),\left(1, -1\right),\left(1, 0\right)\right\}\right\}$ \\
\end{tabular}$ \\ \hline
$14$ & $\begin{tabular}{l}
$\left\{ -x_{0}^{2} \, | \, x_{0}\in \mathbb{Z} \, / \,\left\{-1,1\right\}\right\}$ \\
$\left\{ x_{0}^{2} \, | \, x_{0}\in \mathbb{Z} \, / \,\left\{-1,1\right\}\right\}$ \\
\end{tabular}$ \\ \hline
$15$ & $\begin{tabular}{l}
$\left\{ -x_{0}^{2} \, | \, x_{0}\in \mathbb{Z} \, / \,\left\{-1,1\right\}\right\}$ \\
$\left\{ x_{0}^{2} \, | \, x_{0}\in \mathbb{Z} \, / \,\left\{-1,1\right\}\right\}$ \\
\end{tabular}$ \\ \hline
$16$ & $\begin{tabular}{l}
$\left\{ -{\left(x_{0} + x_{1}\right)} x_{0} - x_{1}^{2} \, | \, (x_{0},x_{1})\in \mathbb{Z} \times \mathbb{Z} \, / \,\left\{\left(-1, 0\right),\left(-1, 1\right),\left(0, -1\right),\left(0, 1\right),\left(1, -1\right),\left(1, 0\right)\right\}\right\}$ \\
$\left\{ x_{0}^{2} + {\left(x_{0} + x_{1}\right)} x_{1} \, | \, (x_{0},x_{1})\in \mathbb{Z} \times \mathbb{Z} \, / \,\left\{\left(-1, 0\right),\left(-1, 1\right),\left(0, -1\right),\left(0, 1\right),\left(1, -1\right),\left(1, 0\right)\right\}\right\}$ \\
\end{tabular}$ \\ \hline
$17$ & $\begin{tabular}{l}
$ \left\{ 3 \, x_{0}^{2} \, | \, x_{0}\in \mathbb{Z} \right\} $ \\
$ \left\{ -\frac{3}{4} \, x_{0}^{2} \, | \, x_{0}\in \mathbb{Z} \right\} $ \\
$ \left\{ -3 \, x_{0}^{2} \, | \, x_{0}\in \mathbb{Z} \right\} $ \\
$\left\{ x_{0}^{2} \, | \, x_{0}\in \mathbb{Z} \, / \,\left\{-1,1\right\}\right\}$ \\
$ \left\{ \frac{3}{4} \, x_{0}^{2} \, | \, x_{0}\in \mathbb{Z} \right\} $ \\
$\left\{ -x_{0}^{2} \, | \, x_{0}\in \mathbb{Z} \, / \,\left\{-1,1\right\}\right\}$ \\
\end{tabular}$ \\ \hline
\end{tabular}

\end{scriptsize}

\footnotetext[1]{A positional number of the group in ITA}
\footnotetext[2]{The explicit forms were generated by \url{https://github.com/davendiy/cryst-automata}}

\newpage

\begin{scriptsize}
\begin{table}
\centering
\begin{tabular}{|l|c|c|c|c|c|}
\hline
\textbf{group} & \textbf{minimal alphabet} & \textbf{gens amount} & \textbf{semi-direct} & \textbf{determinant} & \textbf{cond.2} \\
\hline
Group 1 p1    & \textbf{2}     & 2 & + &  &  \\
Group 2 p2    & \textbf{2}     & 3 & + &  &  \\
Group 3 p1m1  & \textbf{4}     & 3 & + & $x_0x_1$ &  \\
Group 4 p1g1  & \textbf{6}     & 2 &   & $x_0x_1$ & $x_1$ odd \\
Group 5 c1m1  & \textbf{4}     & 3 & + & $\left(x_{0} - 2x_{1}\right)x_{0}$ &  \\
Group 6 p2mm  & \textbf{2}     & 4 & + & $-x_0x_1$ &  \\
Group 7 p2mg  & \textbf{6}     & 3 &   & $x_0x_1$ & $x_0$ odd \\
Group 8 p2gg  & \textbf{3}     & 4 &   & $x_0x_1$ & $x_0, x_1$ odd \\
Group 9 c2mm  & \textbf{3}     & 4 & + & $x_{1}(2x_{0} - x_{1})$ &  \\
Group 10 p4   & \textbf{2}     & 4 & + & $x_0^2 + x_1^2$ &  \\
Group 11 p4mm & \textbf{2}     & 5 & + & $-2x_0^2$ &  \\
Group 12 p4gm & \textbf{9}     & 5 &   & $x_0^2$ & $x_0$ odd \\
Group 13 p3   & \textbf{3}     & 3 & + & $x_0^2 + x_0x_1 + x_1^2$ &  \\
Group 14 p3m1 & \textbf{4}     & 4 & + & $x_0^2$ &  \\
Group 15 p31m & \textbf{4}     & 4 & + & $x_0^2$ &  \\
Group 16 p6   & \textbf{3}     & 4 & + & $x_0^2 + x_0x_1 + x_1^2$ &  \\
Group 17 p6mm & \textbf{3}     & 5 & + & $\tfrac{3}{4}x_0^2$ & $x_0$ is even \\
\hline
\end{tabular}
\caption{Summary of groups and their algebraic properties (without binary alphabet notes)}
\label{tab:groups}
\end{table}
\end{scriptsize}


\begin{thebibliography}{20}
  \bibitem{Nekrashevych:Self-similar} V. Nekrashevych. {\it Self-similar groups},
{Mathematical Surveys and Monographs}, Vol.117 (American Mathematical Society, Providence, 2005).

  \bibitem{Szcz:CrystBook} Andrzej Szczepanski. Geometry of crystallographic groups, volume 4
of Algebra and Discrete Mathematics. World Scientific, 2012.

  \bibitem{CrystTables} International Tables for Crystallography (2016). Volume A, Space-group symmetry.

\end{thebibliography}

\end{document}
